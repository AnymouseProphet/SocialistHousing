\section{Residential Hoarding Tax}

\epigraph{``20\% of all homes are sold to investors, who will hold most of them off the market.\\[1\baselineskip] That's not the only reason for the extreme housing shortage in the US, but it's a major factor.''}{\textit{@AGirlJustKnows \\ Twitter---May 23, 2021}}

\noindent In January of 2017 it was estimated there were \num[group-separator={,}]{553742} homeless people in the United States\footurl{https://bit.ly/3tOD5JV}. That is over half a million people. In 2019 that number had increased to \num[group-separator={,}]{567715} homeless people in the United States\footurl{https://bit.ly/3htZRUQ}. Note that number is from \emph{before} all of the evictions that took place as a result of the \covid{} pandemic.

There are far more vacant homes in the United States than there are people who need homes. Granted, many of the vacant homes are not located where the people are, but it is very clear that the housing market is not a fair market. It does not operate on the premise of supply and demand. The supply far exceeds the demand yet housing costs continue to increase at a rate that far outpaces wages. That is neither a free nor a fair market.

Just like health care, the housing market is an exploitative predatory market without effective government oversight and it is a market designed specifically to transfer wealth from the poor to the rich. This \textbf{\emph{must}} be fixed.

To the proponents of `free market capitalism': It is literally \emph{impossible} to actually have a fair market without checks and balances that are enforced. Fair markets simply do not exist without regulation and enforcement of those regulations. To believe otherwise is as logical as believing in rainbow unicorns or perpetual motion machines or a philosopher's stone that can turn lead into gold. If you want a fair market, regulation of that market with checks and balances is the only way to obtain it.

This is a plan to add some proper checks and balances to the housing market. This plan may not be enough to fix the housing market, but it is a start.

Politicians and lawyers probably need to radically clean up the wording.

\subsection{Definitions}

\begin{description}
  \item[Single-Unit Residence] \hfill \\ A residence intended as a single residence that can be sold as a single residence. A house or condominium unit would be typical examples.
  \item[Multi-Residence Unit] \hfill \\ A residence intended as a single residence but is part of a larger structure containing property that is not part of the residence itself but is part of the same legal property. An apartment in an apartment building or a duplex would be typical examples.
  \item[Primary Residence] \hfill \\ The address of residence for a legal adult that qualifies under voting laws as the legal residence for voting purposes or if the person is not registered to vote, the address that would qualify under tax laws as the legal residence for tax purposes. Whether the resident owns the residence or not is not relevant.
  \item[Asset Residence] \hfill \\ A residence that is owned by a person, investment group, or corporation that is not the primary residence of the owner.
  \item[Residential Area] \hfill \\ A building or neighborhood that serves as housing within 35 miles of a public \emph{or private} school that serves children who fall within the K--12 education range. This may differ from local zoning laws but defining it prevents local zoning laws from being re-written to allow predatory hoarding to continue.
  \item[Hoarded Residence] \hfill \\ An `Asset Residence' in a `Residential Area' that is \emph{not serving} as the `Primary Residence' for an occupant for more than 215 consecutive days (just over seven months). The 215 days do not need to fall within the same calendar year. When they span from one calendar year to the next, the residence is considered a `Hoarded Residence' during the calendar year day 215 falls in. Note that any `Asset Residence' that \emph{is not in a `Residential Area'} is not to be classified as a `Hoarded Residence'. An example of a residence that would not be a hoarded residence even if not a `Primary Residence' would be a cabin in the mountains far from urban areas. And no, I do not own one and likely never will.
\end{description}

\subsection{The Hoarder's Tax}
My desire is to see this plan implemented at the national level. However I would encourage state legislatures to implement this at the state level. This is a property tax and property taxes are traditionally not federal. If a property tax legally can not be done at the federal level for some constitutional reason then this needs to be done at the state level, but regardless, it needs to be done.

For a `Single Unit Residence' that is a `Hoarded Residence' a tax of 15\% of the value of the residence will be levied \emph{in addition} to the standard property tax.

For a `Multi-Residence Unit' that is a `Hoarded Residence' a tax of 20\% of the expected yearly rent if the residence was occupied as a `Primary Residence' will be levied \emph{in addition} to the standard property tax.

Rental agreements between a landlord and a tenant will have to specify whether or not the unit is being rented as a `Primary Residence' so that the landlord knows whether or not they need to pay the Hoarder's Tax and can charge the tenant more accordingly.

The taxes levied for a `Hoarded Residence' will go to funding low-income housing. This includes but not exclusively the public works project listed in the `House Retrofitting' (page~\pageref{retrofit}) section of this manifesto.

Properties seized due to a failure to pay taxes \emph{may} be seized and retrofitted for low-income housing as described in the `House Retrofitting' (page~\pageref{retrofit}) section of this manifesto.

\subsubsection{Unfit for Human Housing Exemption}

There are cases where a housing unit is not fit for human housing and can not legally or morally be rented out as a residence.

The Hoarder's Tax \emph{must} take this into consideration.

To start with, when it is deemed the unit can be brought up to livable conditions for a reasonable cost, the owner of the property should be allowed to file for a \textbf{one-time} two-year exemption from the Hoarder's Tax to give them time to bring it up to livable conditions before they are charged for hoarding it.

In the event the owner of the property themselves qualify as economically poor, the owner of the property should be allowed to file for a five-year exemption from the Hoarder's Tax and they should be allowed to re-apply at the end of that five year period.

Additionally, there should be a low-interest loan program facilitated through Post Office Banking (see page~\pageref{USPS}) that can help fund necessary repairs to bring a residence back into a livable condition---presuming the property itself qualifies as collateral for such a loan. Private banks of course could offer their own loans, but by having to compete with USPS banking, they would actually have to offer a fair deal to get the business.

In cases where a housing unit can not be brought up to conditions allowing it to serve as a residence then the housing unit \emph{should} be demolished \emph{unless} it qualifies as a historic building that should be preserved for cultural and/or other historical reasons. Demolishing housing that can not be brought up to livable standards is important to prevent undocumented slumlord rentals in a `Residential Area'.

When a housing property \emph{in a neighborhood} can not be brought to livable conditions at a reasonable cost, I highly recommend that eminent domain be considered so that low-income housing can replace the house rather than expensive housing that risks gentrifying the neighborhood. In some cases, turning the property into a community garden should be considered. This could potentially provide quality fresh vegetables for a community located in a food desert where quality fresh vegetables are not readily available.

In the wake of the pandemic it may be a good idea for the local government to have some properties within poorer neighborhoods that can serve as emergency clinics if and when the next outbreak does occur.

Please note I only mention eminent domain in the context of a `Hoarded Residence' that can not be brought up to livable standards. If it is the primary residence of the property owner then the property owner clearly needs assistance.

\subsubsection{Ghost Town Exemption}

When a major industry leaves an area, the result often can be an excess of housing compared to the number of people who still reside there. When this happens, sometimes there are still enough residents left to warrant a school so the area still would qualify as a `Residential Area' by my definition.

First of all, when it looks like this scenario \emph{is going to} happen but has not yet happened, government should \emph{preemptively} act to help the people. This could be incentives to bring a new industry to the area or it could be financial help to help with relocation of those who will be out of work shortly. That is what government is for.

If government did its job in these cases more often, I believe there would be less resentment towards environmental causes such as preserving old-growth forests or the halting of fracking and coal mining.

Secondly when it does happen and a geographical area has far more housing than demand then the community should be given a `Ghost Town Exemption' so that houses in that area do not count as a `Hoarded Residence' giving the owner ample time to sell if they so choose at a fair price without the looming pressure of the Hoarder's Tax if it does not sell.

The community can be re-evaluated every three years or so to determine if it still qualifies for the `Ghost Town Exemption'.

The actual qualifications for the `Ghost Town Exemption' should be hashed out by politicians with input from the people likely to be impacted. It would not surprise me if the qualifications need to be somewhat fluid. Compassion very often does need to be fluid.

\subsubsection{Vacation Rental Community Exemption}

There are some businesses and partnerships of individuals that own adjacent property collectives \emph{specifically} for the purposes of vacation rentals. When these properties are not part of a neighborhood with homes used as a `Primary Residence' with a noted exception of the business owner or employees of the business then these vacation rentals can be exempt from the Hoarder's Tax even though they could be sold off or rented as `Primary Residence' housing units if the vacation rental business fails.

These vacation rentals \emph{must not} be in a neighborhood or building with residences not owned by the business and the business must be registered as a vacation rental business.

I have never been there, but I believe Windsor Hills in Florida\footurl{https://bit.ly/3uWrgTw} is an example of this kind of vacation rental community.

That vacation rental community appears to exist to serve the financially privileged but on the other hand---that helps keep the cheap motels for poor people visiting that part of Florida somewhat affordable. As it is not a community that exists for the purpose of permanent residences, I would not treat it as residence hoarding.

\subsection{Intended Impact of the Tax}

Currently many property hoarders would rather have a home sit empty than to sell or rent it below a certain price.

This is sometimes the result of collusion with other property owners. It serves both as a way to discriminate against minorities who are less likely to be able to afford the asking price and it serves to keep the property values in the area artificially high---allowing them to secure funding from banks they can invest at a projected higher return than the interest rates (frequently aided by insider trading) and get a tax break on the mortgage from that loan.

The higher property value they report to lenders doesn't however mean they are paying higher property taxes. They find ways around that\footurl{https://bit.ly/3w7RySP} and often do not get caught.

Frequently the homes are monetized by renting them out for parties, vacation rentals like airbnd\textregistered{}, or even rented out for pornography studios or web-cam studios.

With a stiff tax for hoarding these resources, this type hoarding becomes a lot more expensive \emph{and} funds government assistance for low-income housing when the hoarding continues to take place.

Property owners of residential housing can still choose to monetize their property in a manner other than renting it out as a `Primary Residence' but doing so \emph{must} come at a higher tax cost that helps fund housing for those living in poverty who are directly impacted by the choice of the property owner to monetize the hoarded resource in a manner that keeps property prices artificially high.

Another issue that this would address is those who have two locations they use as residences.

I personally know several people who work in Santa Clara County (the heart of Silicon Valley) making a \emph{lot} of money but they choose to own homes in either Contra Costa County or San Joaquin County because Santa Clara County real estate prices are just insane. Santa Clara County really should have built housing rather than a stadium for a football team that doesn't know how to develop quarterbacks.

To compensate for the very long commutes, these people rent studio or single bedroom apartments in Fremont (Alameda County) where they frequently stay during the work week.

This practice results in gentrification of formerly largely Hispanic areas of Contra Costa and San Joaquin counties while at the same time driving up the demand and thus the cost of studio and single bedroom apartments in Alameda and Santa Clara counties. The poor in both areas are screwed over.

If their primary residence is in Contra Costa County or in San Joaquin County then the pad they are renting in Fremont is not their primary residence. The landlord then has to pay the ``Hoarder's Tax'' which in turn is passed on as higher rent because it is not being rented as a primary residence. The higher rent will discourage and reduce the practice but it will not stop the practice. However where there practice continues, the tax will help fund government assistance for low-income housing.

So-called `Fuck Pads' would also increase in price---as they are not the `Primary Residence' of the husband who is using them to cheat on his wife. The increased cost of that hedonism would be going to a tax used to assist with low-income housing---helping to offset the effect of the increased market price of small apartments that are caused by that kind of hedonism.

The tax may even encourage more companies to further reduce how often employees are expected to come in to work, possibly reducing the resulting carbon footprint.

Another benefit in light of the recent pandemic: Low income employees in Santa Clara County---those most likely to be hit by a disease outbreak or pandemic---may have an easier time finding housing closer to Santa Clara County, both reducing the time they spend commuting (allowing educational development) and reducing the spread of outbreaks in Santa Clara County to surrounding areas, such as we saw with \covid{} where wealthy `tech bros' likely brought it into Silicon Valley from New York, Europe, or \emph{maybe} even Asia and then from there it spread rapidly through poor communities in the rest of the south bay aided by the long public transportation commutes of the poor who work in Silicon Valley.

\subsubsection{Tax Fraud}

Some people will try to get around the ``Hoarder's Tax'' by claiming one address as a primary residence while a spouse claims a different address as a primary residence. In fact a similar type of tax fraud already happens\footurl{https://bit.ly/2RpEkSB}.

The Hoarder's Tax needs to explicitly define this kind of fraud as tax fraud.

\subsection{Financial Burden}

The biggest financial burden of this tax will be upon those who own multiple residences---the financially privileged.

I believe the impact on the poor will be a positive impact---the reduction in the cost of housing. Indeed, that is the explicit purpose of this tax.

When this tax is implemented, the \emph{actual} financial impact on the poor \emph{must} be properly studied to make sure the purpose of this tax is in fact being realized.
