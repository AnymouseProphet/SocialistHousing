\section{The Thirty-Five Percent Plan}
\label{thirtyfive}

My desire is to see this plan implemented at the national level. However I would encourage state legislatures to implement this at the state level. Until voter suppression is truly eliminated, getting sixty votes for this in the United States Senate will be particularly difficult. Even getting fifty votes will be difficult when we have Senators from the Democratic Party who like to prance and dance as they deny us a living wage.

This plan \emph{should} be implemented at the federal level but state legislatures \emph{should not} wait for that happen.

Minimum wage for a geographical area must be high enough such that the median rent for a studio apartment does not exceed 35\% of the gross monthly income of a full-time minimum wage earner, assuming 21 days of eight hours a day.

If we let $M$ represent the median monthly rent for a studio apartment and $W$ represent the minimum hourly wage, the following formula would define the lowest minimum wage could be:

\[
W = M / (21 \times 8 \times 0.35)
\]

So for example, if the median studio apartment rent is \usmoney{1400} per month, minimum wage must be at \emph{least}:

\[
W = \usmoney{1400} / (21 \times 8 \times 0.35) = \SI{23.81}[\$]{\per\hour}
\]

By tying minimum wage to rent, cities are encouraged to include low-income housing in city planning to reduce the cost of doing business in the city---gentrification will become an exceedingly expensive proposition. Businesses that choose not to open locations near low-income housing will pay a penalty of the higher wages their employees need in order to have adequate housing.

The difficult part of this plan is defining the housing market intended to provide low-wage employees to businesses. It is fairly obvious that Santa Cruz is not where low-wage people working in San Francisco are expected to be able to live simply due to distance, but there some cases that are not as obvious. For example, El Cerrito, CA is a small city in Contra Costa County where many who work in Oakland or San Francisco live. Studio apartments in El Cerrito would therefore be part of the calculated median rent for minimum wage in both Oakland and San Francisco. Oakland, CA is also where many people who work in San Francisco live, so studio apartments in Oakland would therefore be part of the calculated median rent for minimum wage in both Oakland and San Francisco. However very very few low-wage workers who live in San Francisco would work in either El Cerrito or Oakland---so studio apartments in San Francisco should probably not be considered in determining the minimum wage of either Oakland or El Cerrito.

Data from federal and state tax returns can be used to determine where low-wage employees in a particular city tend to live allowing for an accurate analysis of where studio apartment rentals should be evaluated to determine the appropriate minimum wage for the city. In some cases, it may be beneficial to view multiple cities as a single geographical region. For example, within west Contra Costa County, the cities of El Cerrito, San Pablo, Richmond, and El Sobrante probably constitute a distinct area that should have the same calculated minimum wage. Those kinds of details should \emph{probably} be worked out at the county level, under guidelines developed either by the state government or the federal government. It might also be a good idea to re-evaluate congressional districts at the same time so that the undemocratic practice of district gerrymandering can be undone.

Of course there may be other factors that require a local area to increase their minimum wage above what that formula specifies. For example, in the San Francisco market they really should take the cost of transportation to San Francisco into account and add a couple dollars an hour to what is calculated. However in El Cerrito that probably is not necessary. The point of the formula is to forbid minimum wage from ever falling \emph{below} that number. Individual areas may have circumstances that justify minimum wage being \emph{above} the number the formula produces.

\subsection{Family Housing Units}

The thirty-five percent plan does not address the cost of housing appropriate for families.

A possible flaw in the thirty-five percent plan as presented so far is the over-creation of studio apartment housing to keep minimum wage low and the under-creation of affordable housing units for families.

The thirty-five percent plan therefore \emph{must} include mandates for appropriate ratios of housing units within a city to reflect family demographics.

With the appropriate ratio of two-, three-, and four-bedroom housing units to studio apartments, the housing market will determine fair prices of those housing units relative to studio apartments and the rent for studio apartments that is determined by the housing market will determine the fair minimum wage for cities where residents of that housing market tend to work.

\subsection{Side Effects and the Threat of Automation}

A very possible side-effect of the thirty-five percent rule is we may see some businesses move out of the higher-rent areas in order to reduce the effective minimum wage that they pay. This will reduce the demand for housing in those areas and reduce the rent in those areas, as well as reducing the commute time needed for their employees. I see this as a positive side effect.

Another possible \emph{negative} side-effect, some businesses will threaten to outsource or automate in response to a higher minimum wage.

Andy Puzder, the former Chief Executive Officer (CEO) of CKE Restaurants, threatened to fully automate if the federal minimum wage is increased. The truth is, those who make such threats will outsource or automate \emph{regardless} of whether or not minimum wage increases simply because that is what greedy capitalists do.

Automation continues to drop in price, reducing both the need for skilled and unskilled labor alike. Automation increases production efficiency and increased production efficiency increases the profit margin. The solution to this problem is not to keep minimum wage low, but rather, to use progressive tax rates on businesses to fund the social safety net needed to make sure every American has the three basic necessities needed to have what our Declaration of Independence declares to be our unalienable rights.

With progressive tax rates, increased efficiency from automation will allow Americans to enjoy a 30 or even 25 hour work week without sacrificing quality of life, allowing all Americans to benefit from the increased productivity of automation rather than just the wealthy few who own the means to production. This in turn will lead to more Americans having the ability to invest in the stock market, increasing the number of Americans who own the means to productions, realizing the vision of Carl Marx without the need for a workers revolution and the high cost of blood involved in a workers revolution. This also avoids the uncertainty of the actual political outcome of the aftermath of a workers revolution.

If the work week is reduced then the formula I presented earlier will have to be adjusted to reflect the shorter work weeks \emph{or preferably} Unconditional Basic Income (UBI) will be needed to compensate for the lost work hours. Note that for UBI, what Andrew Yang proposed is rubbish. UBI needs to stack with with government assistance \emph{regardless} of the type or amount of government assistance. That is what `Unconditional' means. What Yang proposed in his presidential bid had conditions that would harm those with disabilities\footurl{https://bit.ly/3u50Hu4}. Yang's UBI was a libertarian Trojan horse.

Brains before bullets. If we can transition to become a more socialist government peacefully through democracy \emph{without} voter suppression, we should.

Such a peaceful transition was the vision of the Reverend Dr.\ Martin Luther King Jr. It is time we stop just celebrating his birthday and start implementing the dream he had for the people of America.

\subsection{A Message to the \texorpdfstring{`Fiscally Conservative'}{'Fiscally Conservative'}}

A lot of people in my life are either politically conservative and `fiscally conservative' or politically liberal and `fiscally conservative'. Both groups do not like seeing tax money going to pay for social services for `the poor' who they see as lazy.

Pay a fair wage that reflects the housing costs of an area and guess what: Fewer people need social services to survive, resulting in a reduction of government spending on social services. Therefore, if you are fiscally conservative, logic dictates you embrace this idea.

Of course, many just use the term `fiscally conservative' to justify their personal greed and lust for increased profits while people are literally dying from lack of resources. My message for them is not so kind, I will leave it up to the reader to imagine.
