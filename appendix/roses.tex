\section{Bread and Roses}

\begin{quote}
``Not at once; but woman is the mothering element in the world and her vote will go toward helping forward the time when life's Bread: which is home, shelter, and security---and the Roses of life: music, education, nature, and books---shall be the heritage of every child that is born in the country, in the government of which she has a voice.''---\textit{Helen M.\ Todd, 1910}
\end{quote}

It is important to understand where the phrase `Bread and Roses' comes from. The phrase was coined by the woman suffragist Helen Todd, not by James Oppenheim though he is often associated with it. It came from a call for democracy in an era when a woman's voice was suppressed---the Women's Suffrage movement.

Unfortunately in the United States of America---much like the unalienable rights in the Declaration of Independence was largely only applied to white men---the Women's Suffrage movement was \emph{initially} largely only for white women. In the Lawrence Textile Strike of 1912, the `Bread and Roses' phrase was used in the context of low-wage immigrant child- and women-worker rights. For both of those reasons, I have chosen to change the phrase `the rising of the race' to `the rising of the wage' as I initially misheard the reciting of Oppenheim's poem for years.

I expand `Bread and Roses' to include the oppressed working class in general---regardless of gender---but especially minorities who face the most discrimination.

\bigskip

\noindent\underline{\textit{`Bread and Roses' by \textbf{James Oppenheim}---American Magazine, 1911}}

\bigskip

\noindent As we come marching, marching in the beauty of the day,\\
A million darkened kitchens, a thousand mill-lofts gray,\\
Are touched with all the radiance that a sudden sun discloses,\\
For the people hear us singing: ``Bread and Roses! Bread and Roses!''

\medskip

\noindent As we come marching, marching, we battle too for men---\\
For they are women's children, and we mother them again.\\
Our days shall not be sweated from birth until life closes---\\
Hearts starve as well as bodies: give us Bread, but give us Roses.

\medskip

\noindent As we come marching, marching, unnumbered women dead\\
Go crying through our singing their ancient song of Bread;\\
Small art and love and beauty their drudging spirits knew---\\
Yes, it is Bread we fight for---but we fight for Roses, too!

\medskip

\noindent As we come marching, marching, we bring the greater days---\\
The rising of the women means the rising of the wage.\\
No more the drudge and idler---ten that toil where one reposes---\\
But a sharing of life's glories: Bread and Roses! Bread and Roses.
