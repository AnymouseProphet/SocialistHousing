\section{Social Media, Printing, and Donations}

This manifesto exists as a free download from the Internet Archive at the following link:

\bigskip

\url{\canonical}

\bigskip

Please share the link to the canonical version of this manifesto on social media using the following hashtag:

\bigskip

\texttt{\#HousingManifesto}

\subsection{Printing}

The reader is also encouraged to \textbf{print} copies of this manifesto and send them to your representatives in both your state and federal government, along with a personal letter expressing why you believe this plan should be implemented.

A version of the manifesto that differs from the signed canonical version only in that it \emph{lacks} the ISBN number and the digital signature can be found at:

\bigskip

\url{\printversion}

\bigskip

That PDF is perfect for black and white printing.

I recommend a cheap spiral binding. Not only is it cheaper than actual binding, it allows the manifesto to be laid flat for reading. There is no cover art, just ask for transparent plastic cover if a cover is desired.

In order to reduce the cost of printing, this manifesto has been typeset using the U.S.\ letter paper size with only black ink. In an attempt to reduce the paper needed to print this manifesto, this manifesto is typeset for two-sided printing. Your local office supply store should have no problem producing high quality yet cheaply printed versions of this manifesto.

If you can afford to do so, please also print copies of this manifesto to distribute wherever people gather.

Let's fix this country for the benefit of everyone.

\subsection{Twitter}

The author \emph{presently} has a twitter account at:

\bigskip

\url{https://twitter.com/AnymouseProphet}

\bigskip

The author goes through usage spurts, sometimes not tweeting for days or weeks.

\subsection{Mastodon}

The author is on the \url{todon.eu} Mastodon instance at:

\bigskip

\url{https://todon.eu/@AnymouseProphet}

\bigskip

You do not have to be a member of that specific instance to follow the author on Mastodon.

However note that the author goes through usage spurts, sometimes not tooting for days or weeks.

\subsection{Bitcoin}

The author is living in extreme poverty. This is caused by several disabilities that have become markedly worse as a direct result of a lack of access to proper health care. Epilepsy is the primary culprit if curious, but not the only culprit.

With my epilepsy, I sometimes have to suddenly stop what I am doing and lie down and sleep for several hours to \emph{hopefully} avoid a seizure. Both stress and poor-quality lighting increases the odds that this will happen. Some colognes and perfumes people commonly wear at the work place also seem to increase the occurrences but that is anecdotal and not verified by the doctors I can not afford. If a neighbor has used `Roundup\textregistered{} Weed \& Grass Killer' it also seems to increase the occurrences but again that is anecdotal.

Even though this manifesto was created almost exclusively using quality Free Libre Open Source Software (FLOSS) and thus I did not experience the myriad of frustrations that come with commercial software---there were expenses that went into the creation of this manifesto as well as an incredible amount of time and tears.

If you find this manifesto to be worthy of a tip, then I beg you please send a tip to the following Bitcoin address:

\bigskip

\ocr{\small 19rqmw59KZ37wXzD6nkYUrJoZfSRdk8MNj}
\quad
\qrcode[height=5cm,level=Q]{bitcoin:19rqmw59KZ37wXzD6nkYUrJoZfSRdk8MNj}

\bigskip

If you do not know how to send Bitcoin, it can be sent using the \myhref{https://cash.app/}{Cash App} application for Android or Apple phones and tablets.

I will not know who you are, so I will say \textbf{THANK YOU VERY MUCH} right now.
