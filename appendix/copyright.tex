\section{Copyright and Distribution}

This manifesto is \copyright 2021 The Anymouse Prophet and may be freely redistributed under the terms of the Creative Commons Attribution-NonCommercial 4.0 International (CC BY-NC 4.0) license. The text of that license may be found at:

\bigskip

\url{https://creativecommons.org/licenses/by-nc/4.0/}

\bigskip

\noindent The author believes the current length of copyright benefits the wealthy and harms those in poverty. The author believes society at large would benefit if a copyright was only valid for seven years after public publishing.

When the date-stamp on the title page of the canonical version of this manifesto reaches seven years of age the author authorizes free redistribution under the terms of the Creative Commons ``No Rights Reserved'' (CC0) license, effectively \emph{at that time} making the canonical version of this manifesto part of the Public Domain. The text of that license may be found at:

\bigskip

\url{https://creativecommons.org/share-your-work/public-domain/cc0/}

\bigskip

\noindent The canonical version of this manifesto may be retrieved from \url{archive.org} at the following link:

\bigskip

\url{\canonical}

\subsection{ISBN Code}

The ISBN code is only applicable to the canonical version of this publication, the version that (as a PDF file) bears my cryptography signature. You may not use the same ISBN code on an altered version. Hell, even I may not use the same ISBN code on an altered version, no matter how slight the alteration.

You don't want to mess with the ISBN police, they are scary. I mean really scary. Like, Walker Texas Ranger scary\footnote{\url{https://bit.ly/3hvBfLs}}. Okay cereally I have no idea what they would do but those are the rules, even the slightest alteration requires a new ISBN.

%In civilized nations an ISBN code is free but alas---we do not live in a civilized nation. Pay the piper if you need one for an altered version of this manifesto.

%With respect to the LCCN I believe it may be reused as long as the change is superficial rather than a significant content change.

\subsection{Bookstore Sales Exception}

The Creative Commons license specified is a \emph{non-commercial} license.

I specifically give the following exception to bookstores that cater primarily to a Socialist, Communist, and/or Anarchist clientele:

Without and royalties due, you may print and sell physical \emph{unaltered} copies of the canonical version of this manifesto for a profit even before this manifesto reaches seven years of age. You may not however sell digital copies---those can only be provided for free. You also may not sell altered copies.

The `Bookstore Sales Exception' does not apply to wholesalers.

The primary purpose of this exception is to allow this manifesto to be included in `book bundles' that these bookstores sometimes offer. A secondary purpose is to provide access of a printed copy to people who do not have the ability to create a printed copy themselves.

\subsubsection{Cover and ISBN Issue}

Bookstores covered by this exception---at their discretion---may create a cover page with cover art and may reference themselves on that cover page. However, when doing so they \emph{must} use the non-canonical version of the PDF that differs only in lacking the ISBN number and digital signature.

This version of the PDF file without the ISBN or digital signature may be retrieved from:

\bigskip

\url{\printversion}

\bigskip

Bookstores covered by this exception who are using the above linked version may also---at their discretion---create a back page and may use their own ISBN number even listing themselves as the publisher should they so choose to do so.

However, the content itself in that PDF file \emph{including the LCCN} must not be altered.

Any alteration of the content itself and the CC BY-NC 4.0 applies in which case it can not be sold for commercial profit.

\subsection{Github}

The \LaTeX{} source for this manifesto can be found at:

\bigskip

\url{\github} 
