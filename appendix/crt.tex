\section{Critical Race Theory}

I am only mentioning Critical Race Theory (often abbreviated as CRT) because of a recent attempt to ban it in the public education system in some places. It is mentioned in an appendix chapter because I do not feel qualified to give a detailed account of the theory itself but I do feel qualified to talk about the shameful attempts to ban it.

First of all, if you asked me to write a 1500 word essay on Critical Race Theory it would likely take me at least 20--40 hours of research just to write something of the quality of a typical tenth grade essay yet it appears quite clear to me that the vast majority of the people objecting to Critical Race Theory know even less about the theory itself than I do.

Critical Race Theory is not typically taught in public K--12 schools and while it is often taught at the college level, most college students simply do not take a course that covers the topic in any depth. Usually it is presented in something like a Sociology class or a Philosophy class and when presented it usually is not presented in depth in courses that are not part of a Philosophy or Political Science related major study.

Critical Race Theory is one of several different hypothesis that has been developed to explain ethnic related statistical data tendencies. I do not know Critical Race Theory well enough to have an informed academic opinion on whether or not it is the best explanation for the data.

For a simplistic example, just under 60\% of NFL players are Black. This is a statistically significant four times higher than the general population of the United States of America that are black, which is at just under 14\%. What causes such a statistical difference?

One \textbf{\emph{absurd}} postulation I have heard from several White people is that during the enlightenment of Europe, natural selection selected for Europeans who were better at intellectual pursuits while in Africa tribal warfare selected for Africans that are better at physical competition. That postulation is absurd, has no actual data to back it up, and has an extensive amount of data that refutes it but it still is a postulation I have heard from numerous White people who by believing such an absurd postulate clearly did not inherit alleles themselves that are good for intellectual pursuits\ldots{} These people also do not understand how natural selection works but then again most people do not.

A much more likely explanation is that due to systemic racism within the job market of the United States, someone who is Black and wants a career beyond the hardships of the working class does not have as many options as someone who is white and wants a career beyond the hardships of the working class.

Regardless of your ethnic background, if you have athletic aptitude and can make it as a professional athlete, choosing to pursue a career as a football player can be very risky to your physical health. Many of our football greats such Travis Williams, Mike Webster, Junior Seau, and others had very painful lives after football directly due to injuries sustained while playing the sport. If you have other options to make a good financial life without such personal risk there is a good chance you will take one of those other options. If you are Black those other options are simply far less likely to be available.

Despite just under 60\% of NFL players being Black---the positions of Quarterback and Placekicker are still largely dominated by White athletes. How is this explained?

One \emph{possible} explanation is that for the position of Quarterback, as a team leadership position, the largely White owners and coaches---intentionally or not---allow their personal racial biases to enter their decision-making process. Colin Kaepernick may not have been a Russel Wilson but he was a fantastic quarterback that led the San Francisco 49ers to an NFC Championship yet soon after could not get a job even as a backup despite numbers that were better than many White starting quarterbacks. Given the opportunity he very well may have developed into a Russel Wilson caliber quarterback but after barely losing Superbowl \amprom{47} he was not given another chance.

With respect to Placekickers, there I \emph{suspect} the reason has to do with economics. Economically, you are far more likely to be poor if you are Black than if you are White. If you are poor, how do you pay for college? If you have athletic aptitude a sports scholarship is one way but you are far less likely to get such a scholarship as a Placekicker than for a position like Running Back or Defensive Tackle. I \emph{suspect} in college that the position of Placekicker is more likely to go to a player who is not there on a large sports scholarship and thus more likely to be White.

I have no idea if the explanations I just presented here for the racial statistics in the NFL are accurate causes for those statistics nor do I know if the explanations I just presented here are congruent with Critical Race Theory but that is what Critical Race Theory does: It looks at the very real statistical differences from an academic perspective looking for an explanation behind the statistical differences---whether they are football statistics, military personal statistics, crime rate statistics, education level statistics, whatever.

Critical Race Theory studies the society to look for systemic causes behind racial disparity and how we as a society can address those systemic causes so that the vision of the
Reverend Dr.\ Martin Luther King Jr.\ can finally be realized.

Critical Race Theory believes that racial disparity is the result of systemic constructs in our society that both can and should be undone. It is a highly intellectual pursuit and is well beyond the scope of a K--12 education. It is illogical to pass laws forbidding it in a K--12 education both because it simply is not taught in a K--12 education and because there really is nothing about Critical Race Theory that is objectionable unless you happen to be a White Nationalist asshole that likes benefiting from the racial disparity.

What scares me is that if such laws are passed, these laws will be applied with ignorance. For example, the concept of `White Privilege' is a concept that is discussed within Critical Race Theory but the concept of `White Privilege' itself is not Critical Race Theory.

Would a law banning Critical Race Theory in K--12 education mean that any teachers discussing `White Privilege' are reprimanded or fired? I do not have a lot of faith that our public educators will be knowledgeable enough to know when something is Critical Race Theory and when it isn't and that laws forbidding Critical Race Theory will be applied to forbid the discussion of other very important ethnic related topics.

Furthermore, it seems to me that the purpose of education is the pursuit of knowledge. Why on earth would we then ban the pursuit of knowledge? That seems like it is born from a desire for ignorance.

Critical Race Theory is important in understanding housing disparity between ethnicities and in understanding how our society can correct those housing disparities. Do not promote willful ignorance. Do not support any laws that forbid teaching Critical Race Theory---whether at the K--12 education level or at the collegiate education level.

Part of my K--12 education came from Accelerated Christian Education (ACE) where the `Social Studies' text specifically taught:

\begin{quote}
``Although apartheid appears to allow the unfair treatment of blacks, the system has worked in South Africa\ldots{}The government must be responsible to the taxpayers who provide the money that the government spends. Since that is true only taxpayers should be given the privilege of voting\ldots{}The apartheid policy of South Africa is a modern example of this principle.''
\end{quote}

ACE is common curriculum in Evangelical and Fundamentalist schools. Those same Evangelicals and Fundamentalists now objecting to Critical Race Theory never said a peep about that blatant support of racism in their curriculum. I think \amphref{https://bit.ly/3ggPPp9}{John 8:44} applies to them.

\bigskip

I would caution the reader not to assume that everyone they meet who objects to Critical Race Theory has racist motivations for their objections. My personal experience since it has become such a hot topic is that ignorance is behind the objection.

The \emph{real racists} have positions of media influence at places like Fox News or OANN or have leadership positions within their church. The masses of people speaking out against it however are just ignorant tools being manipulated for the racist agenda of those influencers \emph{without actually understanding} they are being manipulated. Ignorance is tough for them to overcome when they have been so consistently told everything not right-wing is `fake news'.

Anyway, do not fear Critical Race Theory. It is an important field of academic research.

